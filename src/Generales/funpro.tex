\documentclass[12pt,a4paper]{article}
\usepackage[T1]{fontenc}
% Agregar la carpeta styles a la ruta de búsqueda de paquetes
\makeatletter
\def\input@path{{../../styles/}}
\makeatother
\usepackage{syllabus}

% --- Información del curso ---
\universidad{PONTIFICIA UNIVERSIDAD CATÓLICA DEL PERÚ}
\facultad{FACULTAD DE CIENCIAS E INGENIERÍA}
\departamento{DEPARTAMENTO DE CIENCIAS DE LA COMPUTACIÓN}
\curso{FUNDAMENTOS DE PROGRAMACIÓN}
\codigo{INF144}
\semestre{2025-1}
\profesor{Dr. Carlos Mendoza}
\creditos{4}
\horas{5}
\prerequisitos{Ninguno}

% --- START OF DOCUMENT ---
\begin{document}

% --- Generar portada con el nuevo estilo ---
\portadaSilabo

% --- I. INFORMACIÓN GENERAL ---
\section{INFORMACIÓN GENERAL}
\begin{tabularx}{\textwidth}{@{}>{\color{pucpGris}\bfseries}l@{\hspace{1em}}X@{}}
    Clave             & : \textcolor{pucpRojo}{\textbf{\@codigo}} \\
    Créditos          & : \@creditos \\
    Tipo              & : Obligatorio \\
    Horas de dictado  & : \\
    \multicolumn{1}{@{}l@{\hspace{2em}}}{Clase} & : 3,00 horas semanales \\
    \multicolumn{1}{@{}l@{\hspace{2em}}}{Laboratorio} & : 2,00 horas semanales \\
    Horario           & : Lunes y Miércoles de 10:00 a 12:00 \\
    Profesor(es)      & : \@profesor \\
    Departamento      & : Ciencias de la Computación \\
    Requisitos        & : \@prerequisitos
\end{tabularx}
\vspace{0.5cm}

% --- II. SUMILLA / DESCRIPCIÓN DEL CURSO ---
\section{SUMILLA}
\begin{tcolorbox}[colback=pucpRojo!5,colframe=pucpRojo,title=\textbf{Descripción General del Curso}]
El curso "Fundamentos de Programación" introduce a los estudiantes en los conceptos básicos y esenciales de la programación de computadoras. Se exploran las herramientas fundamentales para el diseño de algoritmos, como diagramas de flujo y pseudocódigo, y se desarrollan habilidades para la resolución sistemática de problemas mediante el pensamiento algorítmico. El curso cubre operaciones básicas, estructuras de control, tipos de datos fundamentales, y mecanismos de entrada/salida, sentando las bases para cursos más avanzados de la carrera.
\end{tcolorbox}
\vspace{0.5cm}

% --- III. OBJETIVOS DE APRENDIZAJE / COMPETENCIAS ---
\section{OBJETIVOS Y COMPETENCIAS}

\begin{objetivos}
Al finalizar el curso, el estudiante estará en capacidad de:
\begin{itemize}[leftmargin=*]
    \item Comprender la evolución histórica de la programación y su importancia en el contexto actual.
    \item Diseñar algoritmos utilizando diagramas de flujo y pseudocódigo como herramientas de representación.
    \item Implementar soluciones algorítmicas usando tipos de datos fundamentales y operaciones básicas.
    \item Aplicar estructuras de control de secuencia, selección e iteración para resolver problemas.
    \item Desarrollar programas que interactúen con el usuario mediante operaciones de entrada y salida.
    \item Analizar y depurar programas simples identificando errores lógicos y sintácticos.
\end{itemize}
\end{objetivos}

\begin{competencias}
El curso contribuye al desarrollo de las siguientes competencias profesionales:
\begin{itemize}[leftmargin=*]
    \item \textbf{Pensamiento algorítmico:} Capacidad para formular soluciones computacionales paso a paso.
    \item \textbf{Abstracción:} Habilidad para identificar aspectos esenciales de un problema y modelarlos en términos computacionales.
    \item \textbf{Razonamiento lógico:} Destreza para construir secuencias lógicas de instrucciones que resuelvan problemas específicos.
    \item \textbf{Comunicación técnica:} Capacidad para expresar soluciones algorítmicas en forma clara y precisa utilizando representaciones estándar.
\end{itemize}
\end{competencias}
\vspace{0.5cm}

% --- IV. CONTENIDO / PROGRAMA ANALÍTICO ---
\section{CONTENIDO DEL CURSO}

\begin{tcolorbox}[colback=white,colframe=pucpAzul,title=\textbf{UNIDAD 1: Introducción a la Programación}]
\begin{itemize}[leftmargin=*]
    \item \textbf{Tema 1.1:} Introducción e historia (1,00 hora)
    \item \textbf{Tema 1.2:} Diagramas de flujo y Pseudocódigo (3,00 horas)
\end{itemize}
\textbf{Duración total:} 4,00 horas teóricas
\end{tcolorbox}

\begin{tcolorbox}[colback=white,colframe=pucpAzul,title=\textbf{UNIDAD 2: Conceptos Fundamentales}]
\begin{itemize}[leftmargin=*]
    \item \textbf{Tema 2.1:} Operaciones básicas (aritméticas, booleanas) (2,00 horas)
    \item \textbf{Tema 2.2:} Entrada y Salida de datos (con formato) (2,00 horas)
    \item \textbf{Tema 2.3:} Variables y Tipos de Datos (3,00 horas)
\end{itemize}
\textbf{Duración total:} 7,00 horas teóricas
\end{tcolorbox}

\begin{tcolorbox}[colback=white,colframe=pucpAzul,title=\textbf{UNIDAD 3: Estructuras de Control}]
\begin{itemize}[leftmargin=*]
    \item \textbf{Tema 3.1:} Estructuras secuenciales (2,00 horas)
    \item \textbf{Tema 3.2:} Estructuras condicionales (if-else, switch) (3,00 horas)
    \item \textbf{Tema 3.3:} Estructuras iterativas (for, while, do-while) (4,00 horas)
\end{itemize}
\textbf{Duración total:} 9,00 horas teóricas
\end{tcolorbox}

\begin{tcolorbox}[colback=white,colframe=pucpAzul,title=\textbf{UNIDAD 4: Modularización}]
\begin{itemize}[leftmargin=*]
    \item \textbf{Tema 4.1:} Funciones y procedimientos (4,00 horas)
    \item \textbf{Tema 4.2:} Paso de parámetros (3,00 horas)
    \item \textbf{Tema 4.3:} Ámbito de variables (2,00 horas)
\end{itemize}
\textbf{Duración total:} 9,00 horas teóricas
\end{tcolorbox}

\begin{tcolorbox}[colback=white,colframe=pucpAzul,title=\textbf{UNIDAD 5: Introducción a Estructuras de Datos}]
\begin{itemize}[leftmargin=*]
    \item \textbf{Tema 5.1:} Arreglos unidimensionales (3,00 horas)
    \item \textbf{Tema 5.2:} Arreglos multidimensionales (3,00 horas)
    \item \textbf{Tema 5.3:} Introducción a estructuras/registros (2,00 horas)
\end{itemize}
\textbf{Duración total:} 8,00 horas teóricas
\end{tcolorbox}

\begin{center}
\begin{tikzpicture}
% Diagrama visual de la relación entre unidades
\node[draw=pucpAzul, fill=pucpAzul!10, rounded corners, minimum width=4cm, minimum height=1cm] (intro) at (0,0) {Introducción a la Programación};
\node[draw=pucpAzul, fill=pucpAzul!10, rounded corners, minimum width=4cm, minimum height=1cm] (fund) at (0,-1.5) {Conceptos Fundamentales};
\node[draw=pucpAzul, fill=pucpAzul!10, rounded corners, minimum width=4cm, minimum height=1cm] (estr) at (0,-3) {Estructuras de Control};
\node[draw=pucpAzul, fill=pucpAzul!10, rounded corners, minimum width=4cm, minimum height=1cm] (mod) at (0,-4.5) {Modularización};
\node[draw=pucpRojo, fill=pucpRojo!10, rounded corners, minimum width=4cm, minimum height=1cm] (dat) at (0,-6) {Introducción a Estructuras de Datos};

\draw[->, thick, pucpGris] (intro) -- (fund);
\draw[->, thick, pucpGris] (fund) -- (estr);
\draw[->, thick, pucpGris] (estr) -- (mod);
\draw[->, thick, pucpGris] (mod) -- (dat);
\end{tikzpicture}
\end{center}

\vspace{0.5cm}

% --- V. METODOLOGÍA ---
\section{METODOLOGÍA}
\begin{tcolorbox}[colback=pucpGris!5,colframe=pucpGris,title=\textbf{Enfoque Metodológico}]
El curso se desarrolla mediante una combinación de:

\begin{itemize}[leftmargin=*]
    \item \textbf{Clases teóricas:} Presentación de conceptos fundamentales de programación con ejemplos prácticos.
    \item \textbf{Sesiones de laboratorio:} Implementación de algoritmos y resolución de problemas en un entorno de programación.
    \item \textbf{Resolución de problemas:} Ejercicios progresivos que refuerzan los conceptos aprendidos.
    \item \textbf{Proyectos guiados:} Desarrollo de aplicaciones simples que integran los conceptos del curso.
    \item \textbf{Demostraciones:} Ejemplificación de técnicas de depuración y buenas prácticas de programación.
\end{itemize}

\begin{center}
\begin{tikzpicture}
\draw[fill=pucpRojo!10, draw=pucpRojo, thick, rounded corners] (0,0) rectangle (2,1) node[pos=.5] {Conceptos};
\draw[fill=pucpAzul!10, draw=pucpAzul, thick, rounded corners] (3,0) rectangle (5,1) node[pos=.5] {Práctica};
\draw[fill=pucpDorado!10, draw=pucpDorado, thick, rounded corners] (6,0) rectangle (8,1) node[pos=.5] {Aplicación};
\draw[->, thick, color=pucpGris] (2.1,0.5) -- (2.9,0.5);
\draw[->, thick, color=pucpGris] (5.1,0.5) -- (5.9,0.5);
\end{tikzpicture}
\end{center}

Se utilizará la plataforma virtual institucional para distribución de materiales, entregas de trabajos y comunicación fuera del horario de clases.
\end{tcolorbox}
\vspace{0.5cm}

% --- VI. SISTEMA DE EVALUACIÓN ---
\section{SISTEMA DE EVALUACIÓN}

La nota final (NF) se calculará de la siguiente manera:

\tablaEvaluacion{
    Prácticas de Laboratorio (PL) & Promedio de prácticas semanales & 20\% \\
    Prácticas Calificadas (PC) & Promedio de 3 evaluaciones escritas & 15\% \\
    Proyecto de curso (PR) & Desarrollo de aplicación integradora & 15\% \\
    Examen Parcial (EP) & Evaluación escrita (Unidades 1-3) & 25\% \\
    Examen Final (EF) & Evaluación escrita integral & 25\% \\
}

\begin{tcolorbox}[colback=pucpRojo!5,colframe=pucpRojo,title=\textbf{Políticas de Evaluación}]
\begin{itemize}[leftmargin=*]
    \item Para aprobar el curso se requiere una nota final mínima de 11 sobre 20.
    \item Es obligatoria la asistencia a las sesiones de laboratorio.
    \item La presentación del proyecto de curso es requisito indispensable para aprobar.
    \item Los exámenes no rendidos se califican con nota cero (0).
    \item Casos de plagio serán sancionados según el reglamento de disciplina de la universidad.
    \item Las prácticas de laboratorio son evaluadas tanto por funcionalidad como por estilo y calidad de código.
\end{itemize}
\end{tcolorbox}
\vspace{0.5cm}

% --- VII. CRONOGRAMA / CALENDARIO ---
\section{CRONOGRAMA DEL CURSO}

\cronogramaCurso{
1 & Unidad 1: Introducción e historia & Instalación del entorno de desarrollo \\
2 & Unidad 1: Diagramas de flujo y Pseudocódigo & Laboratorio: Diseño de algoritmos simples \\
3 & Unidad 2: Operaciones básicas & \textbf{Práctica Calificada 1} \\
4 & Unidad 2: Entrada y Salida de datos & Laboratorio: Implementación de E/S \\
5 & Unidad 2: Variables y Tipos de Datos & Laboratorio: Manejo de variables y tipos \\
6 & Unidad 3: Estructuras secuenciales & Introducción al proyecto de curso \\
7 & Unidad 3: Estructuras condicionales & \textbf{Práctica Calificada 2} \\
8 & \textbf{EXAMEN PARCIAL} & Evaluación escrita (Unidades 1-3) \\
9 & Unidad 3: Estructuras iterativas & Laboratorio: Ciclos y repeticiones \\
10 & Unidad 4: Funciones y procedimientos & Laboratorio: Implementación de funciones \\
11 & Unidad 4: Paso de parámetros & Avance de proyecto (Primera revisión) \\
12 & Unidad 4: Ámbito de variables & \textbf{Práctica Calificada 3} \\
13 & Unidad 5: Arreglos unidimensionales & Laboratorio: Manipulación de vectores \\
14 & Unidad 5: Arreglos multidimensionales & Laboratorio: Matrices y aplicaciones \\
15 & Unidad 5: Introducción a estructuras/registros & \textbf{Presentación final de proyectos} \\
16 & \textbf{EXAMEN FINAL} & Evaluación escrita integral \\
}

% --- VIII. BIBLIOGRAFÍA ---
\section{BIBLIOGRAFÍA}

\begin{bibliografiaCurso}
    \item Deitel, P. J., \& Deitel, H. M. (2016). \textit{C How to Program} (8th ed.). Pearson.
    \item Stroustrup, B. (2013). \textit{The C++ Programming Language} (4th ed.). Addison-Wesley.
    \item Cormen, T. H., Leiserson, C. E., Rivest, R. L., \& Stein, C. (2009). \textit{Introduction to Algorithms} (3rd ed.). MIT Press.
    \item Sebesta, R. W. (2015). \textit{Concepts of Programming Languages} (11th ed.). Pearson.
    \item Kernighan, B. W., \& Ritchie, D. M. (1988). \textit{The C Programming Language} (2nd ed.). Prentice Hall.
    \item Joyanes Aguilar, L. (2008). \textit{Fundamentos de Programación: Algoritmos, Estructuras de Datos y Objetos} (4ta ed.). McGraw-Hill.
\end{bibliografiaCurso}

% --- IX. ACERCA DEL DOCENTE ---
\section{INFORMACIÓN DEL DOCENTE}

\begin{tcolorbox}[colback=white,colframe=pucpAzul,title=\textbf{Perfil del Docente}]
\begin{minipage}{0.25\textwidth}
  \centering
  % Imagen representativa del docente (silueta genérica)
  \begin{tikzpicture}
    \fill[pucpGris!30] (0,0) circle (1.2);
    \fill[pucpGris!50] (0,-0.2) circle (0.8);
    \fill[pucpGris!30] (0,1.5) circle (0.6);
  \end{tikzpicture}
\end{minipage}%
\begin{minipage}{0.75\textwidth}
  \begin{tabularx}{\textwidth}{>{\color{pucpGris}\bfseries}l X}
    Nombre & Dr. Carlos Mendoza \\
    Correo Electrónico & cmendoza@pucp.edu.pe \\
    Horario de Asesoría & Martes y Jueves 14:00 - 15:00 \\
    Formación & Doctor en Ciencias de la Computación, Universidad de Illinois \\
    Especialización & Lenguajes de programación, algoritmos y educación en computación \\
    Investigación & Métodos efectivos para la enseñanza de programación, lenguajes educativos \\
  \end{tabularx}
\end{minipage}
\end{tcolorbox}
\vspace{0.5cm}

% --- X. POLÍTICA CONTRA EL PLAGIO ---
\section{POLÍTICA INSTITUCIONAL CONTRA EL PLAGIO}

\begin{tcolorbox}[colback=pucpRojo!5,colframe=pucpRojo,title=\textbf{Integridad Académica}]
Para la corrección y evaluación de todos los trabajos del curso se va a tomar en cuenta el debido respeto a los derechos de autor, castigando severamente cualquier indicio de plagio con la nota CERO (00).

Estas medidas serán independientes del proceso administrativo de sanción que la facultad estime conveniente de acuerdo a cada caso en particular. Se recomienda revisar el Reglamento de Disciplina de la PUCP, disponible en la página web institucional.

La originalidad de los trabajos es un valor esencial de la comunidad académica. Se espera que todos los estudiantes citen adecuadamente las fuentes utilizadas y presenten trabajos que reflejen su propio análisis y reflexión.
\end{tcolorbox}

\begin{center}
\begin{tikzpicture}
\node[inner sep=0pt] at (0,0) {\includegraphics[width=3cm]{../../images/PUCP3.png}};
\end{tikzpicture}
\end{center}

\end{document}