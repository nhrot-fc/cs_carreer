\documentclass[12pt,a4paper]{article}
\usepackage{../styles/syllabus}

% --- Información del curso ---
\universidad{NOMBRE DE LA INSTITUCIÓN}
\facultad{NOMBRE DE LA FACULTAD}
\departamento{NOMBRE DEL DEPARTAMENTO}
\curso{NOMBRE DEL CURSO}
\codigo{CÓDIGO DEL CURSO}
\semestre{AÑO-SEMESTRE}

% --- Configurar encabezado ---
\configurarEncabezado

% --- START OF DOCUMENT ---
\begin{document}

% --- Generar portada ---
\portadaSilabo

% --- I. INFORMACIÓN GENERAL ---
\section*{I. INFORMACIÓN GENERAL}
\begin{tabular}{@{} >{\bfseries}l @{\hspace{1em}} l @{}}
    Clave             & : CÓDIGO\_CURSO \\
    Créditos          & : NÚMERO\_CRÉDITOS \\
    Tipo              & : Obligatorio/Electivo \\
    Horas de dictado  & : \\
    \multicolumn{1}{@{}l@{\hspace{2em}}}{Clase} & : XX horas semanales \\
    \multicolumn{1}{@{}l@{\hspace{2em}}}{Práctica} & : XX horas (semanales/quincenales) \\
    \multicolumn{1}{@{}l@{\hspace{2em}}}{Laboratorio} & : XX horas (semanales/quincenales) \\
    Horario           & : DÍA(S) de HH:MM a HH:MM \\
    Profesor(es)      & : NOMBRE(S) DEL PROFESOR(ES) \\
    Departamento      & : NOMBRE\_DEPARTAMENTO (si aplica) \\
    Requisitos        & : CÓDIGO(S)\_REQUISITO(S) (o Ninguno) \\
\end{tabular}
\vspace{0.5cm}

% --- II. SUMILLA / DESCRIPCIÓN DEL CURSO ---
\section*{II. SUMILLA} % Or DESCRIPCIÓN DEL CURSO
\textit{Escriba aquí la sumilla o descripción general del curso. Este es un resumen conciso de los contenidos y propósitos del curso.}
\vspace{0.5cm}

% --- III. OBJETIVOS DE APRENDIZAJE / COMPETENCIAS ---
\section*{III. OBJETIVOS DE APRENDIZAJE} % Or COMPETENCIAS
Al finalizar el curso, el estudiante estará en capacidad de:
\begin{itemize}[leftmargin=*]
    \item \textit{Objetivo de aprendizaje 1.}
    \item \textit{Objetivo de aprendizaje 2.}
    \item \textit{Objetivo de aprendizaje 3.}
    % Añadir más objetivos si es necesario
\end{itemize}
Estos objetivos contribuyen al logro de los siguientes resultados del programa (si aplica):
\begin{itemize}[leftmargin=*]
    \item \textit{Resultado del programa 1.}
    \item \textit{Resultado del programa 2.}
    % Añadir más resultados si es necesario
\end{itemize}
\vspace{0.5cm}

% --- IV. CONTENIDO / PROGRAMA ANALÍTICO ---
\section*{IV. CONTENIDO (o PROGRAMA ANALÍTICO)}
El curso se divide en las siguientes unidades temáticas:

% Ejemplo de Unidad
\subsection*{UNIDAD 1: Nombre de la Unidad 1}
\begin{itemize}[leftmargin=*]
    \item Tema 1.1: Descripción breve del tema.
    \item Tema 1.2: Descripción breve del tema.
    % ... más temas
\end{itemize}
\textbf{Duración aproximada:} XX horas

\subsection*{UNIDAD 2: Nombre de la Unidad 2}
\begin{itemize}[leftmargin=*]
    \item Tema 2.1: Descripción breve del tema.
    \item Tema 2.2: Descripción breve del tema.
    % ... más temas
\end{itemize}
\textbf{Duración aproximada:} XX horas

% Añadir más unidades según sea necesario
\vspace{0.5cm}

% --- V. METODOLOGÍA ---
\section*{V. METODOLOGÍA}
\textit{Describa aquí la metodología de enseñanza-aprendizaje que se utilizará en el curso. Por ejemplo: clases magistrales, aprendizaje basado en proyectos, estudio de casos, laboratorios, etc.}
\vspace{0.5cm}

% --- VI. SISTEMA DE EVALUACIÓN ---
\section*{VI. SISTEMA DE EVALUACIÓN}
La nota final (NF) se calculará de la siguiente manera:

\tablaEvaluacion{
    Práctica Calificada 1 & PC1 & XX\% \\
    Examen Parcial & EP & XX\% \\
    Trabajo Final & TF & XX\% \\
    Participación en Clase & PA & XX\% \\
}

\textit{Detallar aquí cualquier consideración adicional sobre la evaluación, como políticas sobre trabajos grupales, plagio, asistencia, etc.}
\vspace{0.5cm}

% --- VII. CRONOGRAMA / CALENDARIO ---
\section*{VII. CRONOGRAMA} % O CALENDARIO
\begin{longtable}{p{2cm} p{12cm}}
    \toprule
    \textbf{Semana} & \textbf{Unidad, Tema o Actividad} \\
    \midrule
    \endfirsthead
    \toprule
    \textbf{Semana} & \textbf{Unidad, Tema o Actividad} \\
    \midrule
    \endhead
    \bottomrule
    \endfoot
    1 & \textit{Unidad X. Tema Y / Introducción al curso} \\
    2 & \textit{Unidad X. Tema Z} \\
    3 & \textit{Unidad A. Tema B / Laboratorio 1} \\
    % ... continuar hasta la última semana
    8 & \textit{Examen Parcial} \\
    % ...
    16 & \textit{Examen Final / Entrega Proyecto Final} \\
\end{longtable}
\vspace{0.5cm}

% --- VIII. BIBLIOGRAFÍA ---
\section*{VIII. BIBLIOGRAFÍA}
\subsection*{Bibliografía Obligatoria}
\begin{itemize}[leftmargin=*]
    \item Apellido, N. (Año). \textit{Título del Libro en cursiva} (Edición). Ciudad: Editorial.
    % Ejemplo: Coulouris, G., Dollimore, J., Kindberg, T., & Blair, G. (2011). Distributed Systems: Concepts and Design (5th ed.). Pearson.
    \item \textit{Otro recurso obligatorio.}
\end{itemize}

\subsection*{Bibliografía Complementaria}
\begin{itemize}[leftmargin=*]
    \item Apellido, N. (Año). \textit{Título del Artículo}. Nombre de la Revista, Volumen(Número), pp-pp.
    \item \textit{Otro recurso complementario.}
\end{itemize}
\textit{Nota: Se recomienda seguir un estilo de citación consistente (APA, IEEE, etc.).}
\vspace{0.5cm}

% --- IX. ACERCA DEL DOCENTE (Opcional) ---
\section*{IX. ACERCA DEL DOCENTE}
\textit{Breve reseña del profesor o profesores, incluyendo formación, áreas de especialización, y correo electrónico de contacto.}
\begin{itemize}[leftmargin=*]
    \item \textbf{Nombre del Profesor:} [Nombre Completo]
    \item \textbf{Correo Electrónico:} [email@example.com]
    \item \textbf{Horario de Asesoría:} [Día, HH:MM - HH:MM]
\end{itemize}
\vspace{0.5cm}

% --- X. POLÍTICA CONTRA EL PLAGIO (Opcional pero recomendado) ---
\section*{X. POLÍTICA CONTRA EL PLAGIO}
\textit{Incluir aquí la política institucional o del curso respecto al plagio y la honestidad académica.}
% Ejemplo: "Para la corrección y evaluación de todos los trabajos del curso se va a tomar en cuenta el debido respeto a los derechos de autor, castigando severamente cualquier indicio de plagio con la nota CERO (00). Estas medidas serán independientes del proceso administrativo de sanción que la facultad estime conveniente de acuerdo a cada caso en particular."

\end{document}
% --- END OF DOCUMENT ---