\documentclass[12pt,a4paper]{article}
\usepackage[T1]{fontenc}
% Agregar la carpeta styles a la ruta de búsqueda de paquetes
\makeatletter
\def\input@path{{../../styles/}}
\makeatother
\usepackage{syllabus}

% --- Información del curso ---
\universidad{PONTIFICIA UNIVERSIDAD CATÓLICA DEL PERÚ}
\facultad{FACULTAD DE CIENCIAS E INGENIERÍA}
\departamento{DEPARTAMENTO DE INGENIERÍA}
\curso{ORGANIZACIÓN Y ARQUITECTURA DE COMPUTADORAS}
\codigo{INF244}
\semestre{2025-1}
\profesor{Dr. Ricardo Navarro}
\creditos{4}
\horas{6}
\prerequisitos{Circuitos Digitales, Algoritmia y estructura de datos}

% --- START OF DOCUMENT ---
\begin{document}

% --- Generar portada con el nuevo estilo ---
\portadaSilabo

% --- I. INFORMACIÓN GENERAL ---
\section{INFORMACIÓN GENERAL}
\begin{tabularx}{\textwidth}{@{}>{\color{pucpGris}\bfseries}l@{\hspace{1em}}X@{}}
    Clave             & : \textcolor{pucpRojo}{\textbf{\@codigo}} \\
    Créditos          & : \@creditos \\
    Tipo              & : Obligatorio \\
    Horas de dictado  & : \\
    \multicolumn{1}{@{}l@{\hspace{2em}}}{Clase} & : 4,00 horas semanales \\
    \multicolumn{1}{@{}l@{\hspace{2em}}}{Laboratorio} & : 2,00 horas semanales \\
    Horario           & : Martes y Jueves de 10:00 a 12:00 \\
    Profesor(es)      & : \@profesor \\
    Departamento      & : Ingeniería \\
    Requisitos        & : \@prerequisitos
\end{tabularx}
\vspace{0.5cm}

% --- II. SUMILLA / DESCRIPCIÓN DEL CURSO ---
\section{SUMILLA}
\begin{tcolorbox}[colback=pucpRojo!5,colframe=pucpRojo,title=\textbf{Descripción General del Curso}]
El curso "Organización y Arquitectura de Computadoras" introduce a los estudiantes en el diseño y funcionamiento interno de los sistemas computacionales modernos. Se estudian los diferentes modelos arquitectónicos, componentes fundamentales de una computadora, y cómo estos elementos interactúan para ejecutar programas. El curso abarca desde los modelos conceptuales hasta los componentes físicos, incluyendo la Unidad Central de Procesamiento (CPU), memoria, sistemas de entrada/salida y buses del sistema. Se enfatiza en comprender cómo el diseño de hardware impacta en el rendimiento del software y cómo las decisiones arquitectónicas afectan la eficiencia general del sistema.
\end{tcolorbox}
\vspace{0.5cm}

% --- III. OBJETIVOS DE APRENDIZAJE / COMPETENCIAS ---
\section{OBJETIVOS Y COMPETENCIAS}

\begin{objetivos}
Al finalizar el curso, el estudiante estará en capacidad de:
\begin{itemize}[leftmargin=*]
    \item Comprender y diferenciar los modelos arquitectónicos Von Neumann y Harvard, así como sus implicaciones en el diseño de computadoras.
    \item Analizar y describir los componentes internos de la CPU, incluyendo ALU, registros, y unidad de control.
    \item Entender los diferentes conjuntos de instrucciones y modos de direccionamiento utilizados en arquitecturas modernas.
    \item Explicar la jerarquía de memoria, sus principios de funcionamiento y técnicas de optimización.
    \item Analizar la operación de los sistemas de entrada/salida y buses del sistema.
    \item Evaluar el rendimiento de arquitecturas computacionales utilizando métricas adecuadas.
\end{itemize}
\end{objetivos}

\begin{competencias}
El curso contribuye al desarrollo de las siguientes competencias profesionales:
\begin{itemize}[leftmargin=*]
    \item \textbf{Análisis de sistemas:} Capacidad para comprender sistemas complejos descomponiéndolos en sus componentes funcionales.
    \item \textbf{Diseño digital:} Habilidad para aplicar principios de diseño digital en la evaluación de arquitecturas computacionales.
    \item \textbf{Optimización:} Aptitud para identificar cuellos de botella y proponer mejoras en el rendimiento de sistemas.
    \item \textbf{Integración hardware-software:} Comprensión de cómo las decisiones arquitectónicas afectan al desarrollo y ejecución del software.
\end{itemize}
\end{competencias}
\vspace{0.5cm}

% --- IV. CONTENIDO / PROGRAMA ANALÍTICO ---
\section{CONTENIDO DEL CURSO}

\begin{tcolorbox}[colback=white,colframe=pucpAzul,title=\textbf{UNIDAD 1: Introducción a la Arquitectura de Computadoras}]
\begin{itemize}[leftmargin=*]
    \item \textbf{Tema 1.1:} Modelos Von Neumann y Harvard (2,00 horas)
    \item \textbf{Tema 1.2:} Componentes principales (2,00 horas)
\end{itemize}
\textbf{Duración total:} 4,00 horas teóricas \\
\textbf{Laboratorio:} 2,00 horas semanales
\end{tcolorbox}

\begin{tcolorbox}[colback=white,colframe=pucpAzul,title=\textbf{UNIDAD 2: Instrucciones}]
\begin{itemize}[leftmargin=*]
    \item \textbf{Tema 2.1:} Representación y tipos de instrucciones (4,00 horas)
    \item \textbf{Tema 2.2:} Modos de direccionamiento (4,00 horas)
\end{itemize}
\textbf{Duración total:} 8,00 horas teóricas
\end{tcolorbox}

\begin{tcolorbox}[colback=white,colframe=pucpAzul,title=\textbf{UNIDAD 3: CPU (Unidad Central de Procesamiento)}]
\begin{itemize}[leftmargin=*]
    \item \textbf{Tema 3.1:} Componentes funcionales (4,00 horas)
    \item \textbf{Tema 3.2:} Unidad de control (3,00 horas)
    \item \textbf{Tema 3.3:} Pipeline (Segmentación) (3,00 horas)
\end{itemize}
\textbf{Duración total:} 10,00 horas teóricas
\end{tcolorbox}

\begin{tcolorbox}[colback=white,colframe=pucpAzul,title=\textbf{UNIDAD 4: Memoria}]
\begin{itemize}[leftmargin=*]
    \item \textbf{Tema 4.1:} Jerarquía de memoria y localidad (3,00 horas)
    \item \textbf{Tema 4.2:} Caché (3,00 horas)
    \item \textbf{Tema 4.3:} Principal y Virtual (4,00 horas)
\end{itemize}
\textbf{Duración total:} 10,00 horas teóricas
\end{tcolorbox}

\begin{tcolorbox}[colback=white,colframe=pucpAzul,title=\textbf{UNIDAD 5: I/O (Entrada/Salida)}]
\begin{itemize}[leftmargin=*]
    \item \textbf{Tema 5.1:} Módulos y Técnicas (3,00 horas)
    \item \textbf{Tema 5.2:} Arquitecturas paralelas (3,00 horas)
    \item \textbf{Tema 5.3:} Buses (3,00 horas)
\end{itemize}
\textbf{Duración total:} 9,00 horas teóricas
\end{tcolorbox}

\begin{tcolorbox}[colback=white,colframe=pucpAzul,title=\textbf{UNIDAD 6: Rendimiento}]
\begin{itemize}[leftmargin=*]
    \item \textbf{Tema 6.1:} Métricas y evaluación del rendimiento (1,00 hora)
\end{itemize}
\textbf{Duración total:} 1,00 hora teórica
\end{tcolorbox}

\begin{center}
\begin{tikzpicture}
% Diagrama visual de la relación entre unidades
\node[draw=pucpAzul, fill=pucpAzul!10, rounded corners, minimum width=3cm, minimum height=1cm] (intro) at (0,0) {Introducción};

\node[draw=pucpAzul, fill=pucpAzul!10, rounded corners, minimum width=3cm, minimum height=1cm] (instr) at (-4,-2) {Instrucciones};
\node[draw=pucpAzul, fill=pucpAzul!10, rounded corners, minimum width=3cm, minimum height=1cm] (cpu) at (0,-2) {CPU};
\node[draw=pucpAzul, fill=pucpAzul!10, rounded corners, minimum width=3cm, minimum height=1cm] (mem) at (4,-2) {Memoria};

\node[draw=pucpAzul, fill=pucpAzul!10, rounded corners, minimum width=3cm, minimum height=1cm] (io) at (-2,-4) {I/O};
\node[draw=pucpRojo, fill=pucpRojo!10, rounded corners, minimum width=3cm, minimum height=1cm] (rend) at (2,-4) {Rendimiento};

\draw[->, thick, pucpGris] (intro) -- (instr);
\draw[->, thick, pucpGris] (intro) -- (cpu);
\draw[->, thick, pucpGris] (intro) -- (mem);
\draw[->, thick, pucpGris] (instr) -- (cpu);
\draw[->, thick, pucpGris] (cpu) -- (io);
\draw[->, thick, pucpGris] (mem) -- (io);
\draw[->, thick, pucpGris] (cpu) -- (rend);
\draw[->, thick, pucpGris] (mem) -- (rend);
\draw[->, thick, pucpGris] (io) -- (rend);
\end{tikzpicture}
\end{center}

\vspace{0.5cm}

% --- V. METODOLOGÍA ---
\section{METODOLOGÍA}
\begin{tcolorbox}[colback=pucpGris!5,colframe=pucpGris,title=\textbf{Enfoque Metodológico}]
El curso se desarrolla mediante una combinación de:

\begin{itemize}[leftmargin=*]
    \item \textbf{Clases teóricas:} Exposición de conceptos fundamentales de arquitectura de computadoras.
    \item \textbf{Sesiones de laboratorio:} Implementación práctica de conceptos utilizando simuladores y herramientas de diseño digital.
    \item \textbf{Resolución de problemas:} Análisis y solución de casos relacionados con el rendimiento y diseño de arquitecturas.
    \item \textbf{Proyectos prácticos:} Diseño de componentes simples de arquitectura e implementación en herramientas especializadas.
    \item \textbf{Discusiones dirigidas:} Análisis de arquitecturas comerciales y tendencias actuales.
\end{itemize}

\begin{center}
\begin{tikzpicture}
\draw[fill=pucpRojo!10, draw=pucpRojo, thick, rounded corners] (0,0) rectangle (2,1) node[pos=.5] {Fundamentos};
\draw[fill=pucpAzul!10, draw=pucpAzul, thick, rounded corners] (3,0) rectangle (5,1) node[pos=.5] {Simulación};
\draw[fill=pucpDorado!10, draw=pucpDorado, thick, rounded corners] (6,0) rectangle (8,1) node[pos=.5] {Evaluación};
\draw[->, thick, color=pucpGris] (2.1,0.5) -- (2.9,0.5);
\draw[->, thick, color=pucpGris] (5.1,0.5) -- (5.9,0.5);
\end{tikzpicture}
\end{center}

Se utilizará la plataforma virtual institucional para distribución de materiales, entregas de trabajos y comunicación fuera del horario de clases.
\end{tcolorbox}
\vspace{0.5cm}

% --- VI. SISTEMA DE EVALUACIÓN ---
\section{SISTEMA DE EVALUACIÓN}

La nota final (NF) se calculará de la siguiente manera:

\tablaEvaluacion{
    Laboratorios (L) & Promedio de prácticas de laboratorio & 20\% \\
    Prácticas Calificadas (PC) & Promedio de 3 evaluaciones escritas & 25\% \\
    Examen Parcial (EP) & Evaluación escrita (Unidades 1-3) & 25\% \\
    Examen Final (EF) & Evaluación escrita (Todas las unidades) & 30\% \\
}

\begin{tcolorbox}[colback=pucpRojo!5,colframe=pucpRojo,title=\textbf{Políticas de Evaluación}]
\begin{itemize}[leftmargin=*]
    \item Para aprobar el curso se requiere una nota final mínima de 11 sobre 20.
    \item La asistencia a las sesiones de laboratorio es obligatoria.
    \item La entrega tardía de informes de laboratorio tendrá una penalización de 2 puntos por día de retraso.
    \item Los exámenes no rendidos se califican con nota cero (0).
    \item Casos de plagio serán sancionados según el reglamento de disciplina de la universidad.
\end{itemize}
\end{tcolorbox}
\vspace{0.5cm}

% --- VII. CRONOGRAMA / CALENDARIO ---
\section{CRONOGRAMA DEL CURSO}

\cronogramaCurso{
1 & Unidad 1: Modelos Von Neumann y Harvard & Laboratorio: Introducción a los simuladores \\
2 & Unidad 1: Componentes principales & Laboratorio: Análisis de arquitecturas básicas \\
3 & Unidad 2: Representación y tipos de instrucciones (parte 1) & \textbf{Práctica Calificada 1} \\
4 & Unidad 2: Representación y tipos de instrucciones (parte 2) & Laboratorio: Implementación de instrucciones básicas \\
5 & Unidad 2: Modos de direccionamiento & Laboratorio: Análisis de modos de direccionamiento \\
6 & Unidad 3: Componentes funcionales de la CPU & Laboratorio: Diseño de ALU simple \\
7 & Unidad 3: Unidad de control & \textbf{Práctica Calificada 2} \\
8 & \textbf{EXAMEN PARCIAL} & Evaluación escrita \\
9 & Unidad 3: Pipeline (Segmentación) & Laboratorio: Simulación de pipeline \\
10 & Unidad 4: Jerarquía de memoria y localidad & Laboratorio: Análisis de patrones de acceso a memoria \\
11 & Unidad 4: Caché & Laboratorio: Simulación de memorias caché \\
12 & Unidad 4: Memoria Principal y Virtual & \textbf{Práctica Calificada 3} \\
13 & Unidad 5: Módulos y Técnicas de E/S & Laboratorio: Interfaces de E/S \\
14 & Unidad 5: Arquitecturas paralelas y Buses & Laboratorio: Simulación de buses y protocolos \\
15 & Unidad 6: Métricas y evaluación del rendimiento & Repaso y resolución de consultas \\
16 & \textbf{EXAMEN FINAL} & Evaluación escrita \\
}

% --- VIII. BIBLIOGRAFÍA ---
\section{BIBLIOGRAFÍA}

\begin{bibliografiaCurso}
    \item Patterson, D. A., \& Hennessy, J. L. (2017). \textit{Computer Organization and Design: The Hardware/Software Interface} (6th ed.). Morgan Kaufmann.
    \item Stallings, W. (2016). \textit{Computer Organization and Architecture: Designing for Performance} (10th ed.). Pearson.
    \item Tanenbaum, A. S. (2016). \textit{Structured Computer Organization} (6th ed.). Pearson.
    \item Hamacher, C., Vranesic, Z., \& Zaky, S. (2011). \textit{Computer Organization} (6th ed.). McGraw-Hill.
    \item Harris, D., \& Harris, S. (2015). \textit{Digital Design and Computer Architecture} (2nd ed.). Morgan Kaufmann.
\end{bibliografiaCurso}

% --- IX. ACERCA DEL DOCENTE ---
\section{INFORMACIÓN DEL DOCENTE}

\begin{tcolorbox}[colback=white,colframe=pucpAzul,title=\textbf{Perfil del Docente}]
\begin{minipage}{0.25\textwidth}
  \centering
  % Imagen representativa del docente (silueta genérica)
  \begin{tikzpicture}
    \fill[pucpGris!30] (0,0) circle (1.2);
    \fill[pucpGris!50] (0,-0.2) circle (0.8);
    \fill[pucpGris!30] (0,1.5) circle (0.6);
  \end{tikzpicture}
\end{minipage}%
\begin{minipage}{0.75\textwidth}
  \begin{tabularx}{\textwidth}{>{\color{pucpGris}\bfseries}l X}
    Nombre & Dr. Ricardo Navarro \\
    Correo Electrónico & rnavarro@pucp.edu.pe \\
    Horario de Asesoría & Miércoles 14:00 - 16:00 \\
    Formación & Doctor en Ingeniería de Computadores, Universidad Politécnica de Madrid \\
    Especialización & Arquitecturas avanzadas y sistemas empotrados \\
    Investigación & Arquitecturas de alto rendimiento, sistemas embebidos y computación paralela \\
  \end{tabularx}
\end{minipage}
\end{tcolorbox}
\vspace{0.5cm}

% --- X. POLÍTICA CONTRA EL PLAGIO ---
\section{POLÍTICA INSTITUCIONAL CONTRA EL PLAGIO}

\begin{tcolorbox}[colback=pucpRojo!5,colframe=pucpRojo,title=\textbf{Integridad Académica}]
Para la corrección y evaluación de todos los trabajos del curso se va a tomar en cuenta el debido respeto a los derechos de autor, castigando severamente cualquier indicio de plagio con la nota CERO (00).

Estas medidas serán independientes del proceso administrativo de sanción que la facultad estime conveniente de acuerdo a cada caso en particular. Se recomienda revisar el Reglamento de Disciplina de la PUCP, disponible en la página web institucional.

La originalidad de los trabajos es un valor esencial de la comunidad académica. Se espera que todos los estudiantes citen adecuadamente las fuentes utilizadas y presenten trabajos que reflejen su propio análisis y reflexión.
\end{tcolorbox}

\begin{center}
\begin{tikzpicture}
\node[inner sep=0pt] at (0,0) {\includegraphics[width=3cm]{../../images/PUCP3.png}};
\end{tikzpicture}
\end{center}

\end{document}