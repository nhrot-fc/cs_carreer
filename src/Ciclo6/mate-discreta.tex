\documentclass[12pt,a4paper]{article}
\usepackage[T1]{fontenc}
% Agregar la carpeta styles a la ruta de búsqueda de paquetes
\makeatletter
\def\input@path{{../../styles/}}
\makeatother
\usepackage{syllabus}

% --- Información del curso ---
\universidad{PONTIFICIA UNIVERSIDAD CATÓLICA DEL PERÚ}
\facultad{FACULTAD DE CIENCIAS E INGENIERÍA}
\departamento{DEPARTAMENTO DE CIENCIAS}
\curso{MATEMÁTICAS DISCRETAS Y LÓGICA MATEMÁTICA}
\codigo{MAT281}
\semestre{2025-1}
\profesor{Dra. María Gonzales}
\creditos{4}
\horas{6}
\prerequisitos{Estructuras Discretas, Lógica Matemática}

% --- START OF DOCUMENT ---
\begin{document}

% --- Generar portada con el nuevo estilo ---
\portadaSilabo

% --- I. INFORMACIÓN GENERAL ---
\section{INFORMACIÓN GENERAL}
\begin{tabularx}{\textwidth}{@{}>{\color{pucpGris}\bfseries}l@{\hspace{1em}}X@{}}
    Clave             & : \textcolor{pucpRojo}{\textbf{\@codigo}} \\
    Créditos          & : \@creditos \\
    Tipo              & : Obligatorio \\
    Horas de dictado  & : \\
    \multicolumn{1}{@{}l@{\hspace{2em}}}{Clase} & : 4,00 horas semanales \\
    \multicolumn{1}{@{}l@{\hspace{2em}}}{Prácticas} & : 2,00 horas semanales \\
    Horario           & : Lunes y Jueves de 14:00 a 16:00 \\
    Profesor(es)      & : \@profesor \\
    Departamento      & : Ciencias \\
    Requisitos        & : \@prerequisitos
\end{tabularx}
\vspace{0.5cm}

% --- II. SUMILLA / DESCRIPCIÓN DEL CURSO ---
\section{SUMILLA}
\begin{tcolorbox}[colback=pucpRojo!5,colframe=pucpRojo,title=\textbf{Descripción General del Curso}]
El curso "Matemáticas Discretas y Lógica Matemática" es un curso de nivel avanzado diseñado para proporcionar a los estudiantes una comprensión profunda de temas fundamentales y avanzados en lógica matemática y diversas áreas de la matemática discreta que son cruciales en la informática teórica. El curso explora los fundamentos de la lógica de primer orden, sus teoremas más importantes y sus limitaciones inherentes, como la indecidibilidad. Además, introduce estructuras algebraicas, teoría de códigos y teoría de dominios, herramientas matemáticas esenciales para el modelado y análisis en ciencias de la computación.
\end{tcolorbox}
\vspace{0.5cm}

% --- III. OBJETIVOS DE APRENDIZAJE / COMPETENCIAS ---
\section{OBJETIVOS Y COMPETENCIAS}

\begin{objetivos}
Al finalizar el curso, el estudiante estará en capacidad de:
\begin{itemize}[leftmargin=*]
    \item Dominar los fundamentos avanzados de la lógica de primer orden, incluyendo sintaxis, semántica, sistemas formales, teorías y modelos.
    \item Comprender y aplicar teoremas fundamentales de la lógica como el Teorema de Completitud de Gödel, el Teorema de Compacidad y los Teoremas de Löwenheim-Skolem.
    \item Analizar los límites del poder expresivo y deductivo de los sistemas formales, incluyendo los Teoremas de Incompletitud de Gödel y el concepto de indecidibilidad.
    \item Manejar extensiones y aplicaciones de la lógica de primer orden, como lógicas no clásicas y su uso en la especificación y verificación de sistemas.
    \item Aplicar conceptos avanzados de estructuras algebraicas como grupos, anillos, cuerpos, retículos y álgebras de Boole en contextos computacionales.
    \item Comprender los fundamentos de la teoría de códigos correctores de errores y su importancia en la transmisión y almacenamiento de datos.
    \item Analizar y aplicar conceptos de la teoría de dominios en la semántica de lenguajes de programación.
\end{itemize}
\end{objetivos}

\begin{competencias}
El curso contribuye al desarrollo de las siguientes competencias profesionales:
\begin{itemize}[leftmargin=*]
    \item \textbf{Razonamiento formal:} Capacidad para construir y evaluar argumentos lógicos rigurosos.
    \item \textbf{Pensamiento abstracto:} Habilidad para manejar conceptos abstractos y estructuras matemáticas.
    \item \textbf{Análisis crítico:} Capacidad para identificar limitaciones y alcances de sistemas formales.
    \item \textbf{Modelado matemático:} Destreza para representar problemas computacionales mediante estructuras matemáticas adecuadas.
    \item \textbf{Comunicación formal:} Habilidad para expresar ideas y razonamientos matemáticos con precisión y claridad.
\end{itemize}
\end{competencias}
\vspace{0.5cm}

% --- IV. CONTENIDO / PROGRAMA ANALÍTICO ---
\section{CONTENIDO DEL CURSO}

\begin{tcolorbox}[colback=white,colframe=pucpAzul,title=\textbf{UNIDAD 1: Lógica Matemática Avanzada}]
\begin{itemize}[leftmargin=*]
    \item \textbf{Tema 1.1:} Fundamentos Avanzados de lógica de Primer orden (3,00 horas)
    \item \textbf{Tema 1.2:} Teoremas fundamentales de la lógica (6,00 horas)
    \item \textbf{Tema 1.3:} Límites de la Formalización y decibilidad (6,00 horas)
    \item \textbf{Tema 1.4:} Aplicaciones y extensiones de la lógica (4,00 horas)
\end{itemize}
\textbf{Duración total:} 19,00 horas teóricas \\
\textbf{Prácticas:} 2,00 horas por sesión
\end{tcolorbox}

\begin{tcolorbox}[colback=white,colframe=pucpAzul,title=\textbf{UNIDAD 2: Matemática Discreta Avanzada}]
\begin{itemize}[leftmargin=*]
    \item \textbf{Tema 2.1:} Estructuras Algebraicas fundamentales (6,00 horas)
    \item \textbf{Tema 2.2:} Introducción a la teoría de códigos (9,00 horas)
    \item \textbf{Tema 2.3:} Teoría de dominios y semántica denotacional (8,00 horas)
\end{itemize}
\textbf{Duración total:} 23,00 horas teóricas
\end{tcolorbox}

\begin{center}
\begin{tikzpicture}
% Diagrama visual de la relación entre unidades
\node[draw=pucpAzul, fill=pucpAzul!10, rounded corners, minimum width=6cm, minimum height=1cm] (logica) at (0,0) {Lógica Matemática Avanzada};
\node[draw=pucpRojo, fill=pucpRojo!10, rounded corners, minimum width=6cm, minimum height=1cm] (discreta) at (0,-2) {Matemática Discreta Avanzada};

\draw[->, thick, pucpGris] (logica) -- node[right] {Fundamenta} (discreta);

\node[draw=pucpGris, fill=pucpGris!10, rounded corners, minimum width=2.5cm, minimum height=0.8cm] (est) at (-4,-4) {Estructuras\\Algebraicas};
\node[draw=pucpGris, fill=pucpGris!10, rounded corners, minimum width=2.5cm, minimum height=0.8cm] (cod) at (0,-4) {Teoría de\\Códigos};
\node[draw=pucpGris, fill=pucpGris!10, rounded corners, minimum width=2.5cm, minimum height=0.8cm] (dom) at (4,-4) {Teoría de\\Dominios};

\draw[->, thick, pucpGris] (discreta) -- (est);
\draw[->, thick, pucpGris] (discreta) -- (cod);
\draw[->, thick, pucpGris] (discreta) -- (dom);
\end{tikzpicture}
\end{center}

\vspace{0.5cm}

% --- V. METODOLOGÍA ---
\section{METODOLOGÍA}
\begin{tcolorbox}[colback=pucpGris!5,colframe=pucpGris,title=\textbf{Enfoque Metodológico}]
El curso se desarrolla mediante una combinación de:

\begin{itemize}[leftmargin=*]
    \item \textbf{Clases magistrales:} Presentación formal de conceptos teóricos avanzados con demostraciones rigurosas.
    \item \textbf{Sesiones de prácticas:} Resolución guiada de problemas y ejercicios que refuerzan la comprensión de los conceptos teóricos.
    \item \textbf{Discusiones teóricas:} Análisis de resultados fundamentales y sus implicaciones en la teoría y práctica de la computación.
    \item \textbf{Talleres de aplicación:} Exploración de aplicaciones concretas de los conceptos estudiados en problemas computacionales reales.
    \item \textbf{Presentaciones estudiantiles:} Exposición de temas específicos y resultados de investigación por parte de los estudiantes.
\end{itemize}

\begin{center}
\begin{tikzpicture}
\draw[fill=pucpRojo!10, draw=pucpRojo, thick, rounded corners] (0,0) rectangle (2,1) node[pos=.5] {Teoría};
\draw[fill=pucpAzul!10, draw=pucpAzul, thick, rounded corners] (3,0) rectangle (5,1) node[pos=.5] {Ejercicios};
\draw[fill=pucpDorado!10, draw=pucpDorado, thick, rounded corners] (6,0) rectangle (8,1) node[pos=.5] {Aplicación};
\draw[->, thick, color=pucpGris] (2.1,0.5) -- (2.9,0.5);
\draw[->, thick, color=pucpGris] (5.1,0.5) -- (5.9,0.5);
\end{tikzpicture}
\end{center}

Se utilizará la plataforma virtual institucional para distribución de materiales, entregas de trabajos y comunicación fuera del horario de clases.
\end{tcolorbox}
\vspace{0.5cm}

% --- VI. SISTEMA DE EVALUACIÓN ---
\section{SISTEMA DE EVALUACIÓN}

La nota final (NF) se calculará de la siguiente manera:

\tablaEvaluacion{
    Prácticas Calificadas (PC) & Promedio de 3 evaluaciones escritas & 30\% \\
    Trabajo de Investigación (TI) & Investigación sobre un tema específico & 20\% \\
    Examen Parcial (EP) & Evaluación escrita (Unidad 1) & 25\% \\
    Examen Final (EF) & Evaluación escrita integral & 25\% \\
}

\begin{tcolorbox}[colback=pucpRojo!5,colframe=pucpRojo,title=\textbf{Políticas de Evaluación}]
\begin{itemize}[leftmargin=*]
    \item Para aprobar el curso se requiere una nota final mínima de 11 sobre 20.
    \item Es obligatorio presentarse a todas las evaluaciones programadas.
    \item La no presentación a una evaluación será calificada con nota cero (0), salvo justificación médica documentada.
    \item El trabajo de investigación debe cumplir con los estándares académicos de originalidad y rigor matemático.
    \item Casos de plagio serán sancionados según el reglamento de disciplina de la universidad.
\end{itemize}
\end{tcolorbox}
\vspace{0.5cm}

% --- VII. CRONOGRAMA / CALENDARIO ---
\section{CRONOGRAMA DEL CURSO}

\cronogramaCurso{
1 & Unidad 1: Fundamentos Avanzados de lógica de Primer orden (inicio) & Presentación del curso y formación de equipos \\
2 & Unidad 1: Fundamentos Avanzados de lógica de Primer orden (fin) & Sesión de práctica: Sistemas formales \\
3 & Unidad 1: Teoremas fundamentales de la lógica (parte 1) & \textbf{Práctica Calificada 1} \\
4 & Unidad 1: Teoremas fundamentales de la lógica (parte 2) & Taller: Aplicaciones del teorema de completitud \\
5 & Unidad 1: Teoremas fundamentales de la lógica (parte 3) & Sesión de práctica: Teorema de compacidad \\
6 & Unidad 1: Límites de la Formalización (Teoremas de Gödel) & Discusión dirigida: Implicaciones filosóficas \\
7 & Unidad 1: Indecidibilidad y funciones recursivas & Asignación de temas para trabajo de investigación \\
8 & \textbf{EXAMEN PARCIAL} & Evaluación escrita de la Unidad 1 \\
9 & Unidad 1: Aplicaciones y extensiones de la lógica & Sesión de práctica: Lógicas no clásicas \\
10 & Unidad 2: Estructuras Algebraicas fundamentales (parte 1) & \textbf{Práctica Calificada 2} \\
11 & Unidad 2: Estructuras Algebraicas fundamentales (parte 2) & Taller: Aplicaciones en criptografía \\
12 & Unidad 2: Introducción a la teoría de códigos (parte 1) & Sesión de práctica: Códigos correctores de errores \\
13 & Unidad 2: Introducción a la teoría de códigos (parte 2) & \textbf{Práctica Calificada 3} \\
14 & Unidad 2: Teoría de dominios y semántica denotacional (parte 1) & Entrega de trabajos de investigación \\
15 & Unidad 2: Teoría de dominios y semántica denotacional (parte 2) & Presentaciones de trabajos de investigación \\
16 & \textbf{EXAMEN FINAL} & Evaluación escrita integral \\
}

% --- VIII. BIBLIOGRAFÍA ---
\section{BIBLIOGRAFÍA}

\begin{bibliografiaCurso}
    \item van Dalen, D. (2013). \textit{Logic and Structure} (5th ed.). Springer.
    \item Enderton, H. B. (2001). \textit{A Mathematical Introduction to Logic} (2nd ed.). Academic Press.
    \item Roman, S. (2011). \textit{Advanced Linear Algebra} (3rd ed.). Springer.
    \item MacWilliams, F. J., \& Sloane, N. J. A. (1977). \textit{The Theory of Error-Correcting Codes}. North-Holland.
    \item Winskel, G. (1993). \textit{The Formal Semantics of Programming Languages: An Introduction}. MIT Press.
    \item Boolos, G. S., Burgess, J. P., \& Jeffrey, R. C. (2007). \textit{Computability and Logic} (5th ed.). Cambridge University Press.
    \item Davey, B. A., \& Priestley, H. A. (2002). \textit{Introduction to Lattices and Order} (2nd ed.). Cambridge University Press.
\end{bibliografiaCurso}

% --- IX. ACERCA DEL DOCENTE ---
\section{INFORMACIÓN DEL DOCENTE}

\begin{tcolorbox}[colback=white,colframe=pucpAzul,title=\textbf{Perfil del Docente}]
\begin{minipage}{0.25\textwidth}
  \centering
  % Imagen representativa del docente (silueta genérica)
  \begin{tikzpicture}
    \fill[pucpGris!30] (0,0) circle (1.2);
    \fill[pucpGris!50] (0,-0.2) circle (0.8);
    \fill[pucpGris!30] (0,1.5) circle (0.6);
  \end{tikzpicture}
\end{minipage}%
\begin{minipage}{0.75\textwidth}
  \begin{tabularx}{\textwidth}{>{\color{pucpGris}\bfseries}l X}
    Nombre & Dra. María Gonzales \\
    Correo Electrónico & mgonzales@pucp.edu.pe \\
    Horario de Asesoría & Lunes 10:00 - 12:00 \\
    Formación & Doctora en Matemáticas, Universidad de Cambridge \\
    Especialización & Lógica matemática, teoría de la computabilidad \\
    Investigación & Teoría de modelos, fundamentos matemáticos de la ciencia de la computación \\
  \end{tabularx}
\end{minipage}
\end{tcolorbox}
\vspace{0.5cm}

% --- X. POLÍTICA CONTRA EL PLAGIO ---
\section{POLÍTICA INSTITUCIONAL CONTRA EL PLAGIO}

\begin{tcolorbox}[colback=pucpRojo!5,colframe=pucpRojo,title=\textbf{Integridad Académica}]
Para la corrección y evaluación de todos los trabajos del curso se va a tomar en cuenta el debido respeto a los derechos de autor, castigando severamente cualquier indicio de plagio con la nota CERO (00).

Estas medidas serán independientes del proceso administrativo de sanción que la facultad estime conveniente de acuerdo a cada caso en particular. Se recomienda revisar el Reglamento de Disciplina de la PUCP, disponible en la página web institucional.

La originalidad de los trabajos es un valor esencial de la comunidad académica. Se espera que todos los estudiantes citen adecuadamente las fuentes utilizadas y presenten trabajos que reflejen su propio análisis y reflexión.
\end{tcolorbox}

\begin{center}
\begin{tikzpicture}
\node[inner sep=0pt] at (0,0) {\includegraphics[width=3cm]{../../images/PUCP3.png}};
\end{tikzpicture}
\end{center}

\end{document}