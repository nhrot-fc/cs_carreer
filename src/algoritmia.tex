\documentclass[12pt,a4paper]{article}
\usepackage{../styles/syllabus}

% --- Información del curso ---
\universidad{PONTIFICIA UNIVERSIDAD CATÓLICA DEL PERÚ}
\facultad{FACULTAD DE CIENCIAS}
\departamento{DEPARTAMENTO DE CIENCIAS DE LA COMPUTACIÓN}
\curso{INTRODUCCIÓN A ALGORITMOS}
\codigo{INFXXX}
\semestre{2025-1}

% --- Configurar encabezado ---
\configurarEncabezado

% --- START OF DOCUMENT ---
\begin{document}

% --- Generar portada ---
\portadaSilabo

% --- I. INFORMACIÓN GENERAL ---
\section*{I. INFORMACIÓN GENERAL}
\begin{tabular}{@{} >{\bfseries}l @{\hspace{1em}} l @{}}
    Clave             & : INFXXX \\ % Placeholder
    Créditos          & : 4,00 \\
    Tipo              & : Obligatorio \\
    Horas de dictado  & : \\
    \multicolumn{1}{@{}l@{\hspace{2em}}}{Clase} & : 3,00 horas semanales \\
    \multicolumn{1}{@{}l@{\hspace{2em}}}{Laboratorio} & : 2,00 horas semanales \\
    Horario           & : Lunes y Miércoles de 14:00 a 17:00 \\
    Profesor(es)      & : Dr. Juan Pérez \\
    Departamento      & : Ciencias de la Computación \\
    Requisitos        & : Ninguno \\
\end{tabular}
\vspace{0.5cm}

% --- II. SUMILLA / DESCRIPCIÓN DEL CURSO ---
\section*{II. SUMILLA} % Or DESCRIPCIÓN DEL CURSO
\textit{Este curso proporciona una introducción completa a los conceptos fundamentales del diseño y análisis de algoritmos. Se explorarán técnicas esenciales para la resolución eficiente de problemas computacionales, incluyendo el análisis de eficiencia mediante notación asintótica, estrategias de diseño como divide y vencerás, algoritmos voraces y programación dinámica. Además, se cubrirán estructuras de datos fundamentales y algoritmos de ordenamiento y búsqueda, sentando las bases para el desarrollo de software robusto y eficiente.}
\vspace{0.5cm}

% --- III. OBJETIVOS DE APRENDIZAJE / COMPETENCIAS ---
\section*{III. OBJETIVOS DE APRENDIZAJE} % Or COMPETENCIAS
Al finalizar el curso, el estudiante estará en capacidad de:
\begin{itemize}[leftmargin=*]
    \item Comprender y aplicar la notación asintótica para analizar la eficiencia de los algoritmos.
    \item Diseñar algoritmos utilizando estrategias fundamentales como divide y vencerás, algoritmos voraces y programación dinámica.
    \item Implementar y utilizar eficientemente estructuras de datos esenciales como listas enlazadas, pilas, colas, heaps, hash tables y árboles binarios de búsqueda.
    \item Analizar y aplicar algoritmos de ordenamiento clásicos como Insertion Sort, Bubble Sort, Quick Sort y Merge Sort.
    \item Implementar y analizar algoritmos de búsqueda básicos como Breadth First Search (BFS) y Depth First Search (DFS).
    \item Resolver problemas de optimización aplicando las técnicas algorítmicas adecuadas.
\end{itemize}
\vspace{0.5cm}

% --- IV. CONTENIDO / PROGRAMA ANALÍTICO ---
\section*{IV. CONTENIDO (o PROGRAMA ANALÍTICO)}
El curso se divide en las siguientes unidades temáticas:

\subsection*{UNIDAD 1: Fundamentos}
\begin{itemize}[leftmargin=*]
    \item Introducción (1,00 horas)
    \item Notación asintótica (2,00 horas)
    \item Análisis y diseño de algoritmos (4,00 horas)
    \item Divide y vencerás (4,00 horas)
    \item Teorema Maestro (2,00 horas)
\end{itemize}
\textbf{Total horas Unidad 1:} 13,00 horas

\subsection*{UNIDAD 2: Ordenamiento}
\begin{itemize}[leftmargin=*]
    \item Insertion Sort, Bubble Sort (1,00 horas)
    \item Quick Sort, Merge Sort (2,00 horas)
\end{itemize}
\textbf{Total horas Unidad 2:} 3,00 horas

\subsection*{UNIDAD 3: Estructuras de Datos}
\begin{itemize}[leftmargin=*]
    \item Esenciales (Linked List, Stack, Queue, Heap) (4,00 horas)
    \item Hash Tables (STL) (3,00 horas)
    \item Binary Trees (BST, RBT, AVL) (4,00 horas)
\end{itemize}
\textbf{Total horas Unidad 3:} 11,00 horas

\subsection*{UNIDAD 4: Búsqueda básica}
\begin{itemize}[leftmargin=*]
    \item Breadth First Search (BFS) (3,00 horas)
    \item Depth First Search (DFS) (3,00 horas)
\end{itemize}
\textbf{Total horas Unidad 4:} 6,00 horas

\subsection*{UNIDAD 5: Optimización}
\begin{itemize}[leftmargin=*]
    \item Algoritmos voraces (3,00 horas)
    \item Programación dinámica (6,00 horas)
\end{itemize}
\textbf{Total horas Unidad 5:} 9,00 horas
\vspace{0.5cm}

% --- V. METODOLOGÍA ---
\section*{V. METODOLOGÍA}
\textit{El curso se desarrollará mediante clases teóricas expositivas, donde se presentarán los conceptos y técnicas algorítmicas. Se complementará con sesiones prácticas de laboratorio para la implementación y experimentación de los algoritmos estudiados. Se fomentará la resolución de problemas y el análisis de casos. Se realizarán discusiones grupales y se asignarán trabajos prácticos para reforzar el aprendizaje.}
\vspace{0.5cm}

% --- VI. SISTEMA DE EVALUACIÓN ---
\section*{VI. SISTEMA DE EVALUACIÓN}
La nota final (NF) se calculará de la siguiente manera:

\tablaEvaluacion{
    Prácticas Calificadas (PC) & Promedio de PCs & 30\% \\ % Placeholder
    Laboratorios (Lab) & Promedio de Labs & 20\% \\ % Placeholder
    Examen Parcial (EP) & & 20\% \\ % Placeholder
    Examen Final (EF) & & 30\% \\ % Placeholder
}

\textit{Para aprobar el curso se requiere una nota mínima de 11 sobre 20. Es obligatoria la asistencia a las sesiones de laboratorio. La entrega de trabajos fuera de fecha tendrá una penalización de 2 puntos por día de retraso.}
\vspace{0.5cm}

% --- VII. CRONOGRAMA / CALENDARIO ---
\section*{VII. CRONOGRAMA} % O CALENDARIO
% Ejemplo de tabla de cronograma (Placeholder - ajustar según el calendario real del semestre)
\begin{longtable}{p{2cm} p{12cm}}
    \toprule
    \textbf{Semana} & \textbf{Unidad, Tema o Actividad} \\ %
    \midrule
    \endfirsthead
    \toprule
    \textbf{Semana} & \textbf{Unidad, Tema o Actividad} \\
    \midrule
    \endhead
    \bottomrule
    \endfoot
    1 & Unidad 1: Fundamentos - Introducción, Notación asintótica (inicio) \\ % Placeholder
    2 & Unidad 1: Fundamentos - Notación asintótica (fin), Análisis y diseño de algoritmos (inicio) \\ % Placeholder
    3 & Unidad 1: Fundamentos - Análisis y diseño de algoritmos (fin) \\ % Placeholder
    4 & Unidad 1: Fundamentos - Divide y vencerás (inicio) \\ % Placeholder
    5 & Unidad 1: Fundamentos - Divide y vencerás (fin), Teorema Maestro \\ % Placeholder
    6 & Unidad 2: Ordenamiento - Insertion Sort, Bubble Sort, Quick Sort, Merge Sort \\ % Placeholder
    7 & Unidad 3: Estructuras de Datos - Esenciales (Linked List, Stack, Queue, Heap) \\ % Placeholder
    8 & \textbf{Examen Parcial} \\ % Placeholder
    9 & Unidad 3: Estructuras de Datos - Hash Tables (STL) \\ % Placeholder
    10 & Unidad 3: Estructuras de Datos - Binary Trees (BST, RBT, AVL) (inicio) \\ % Placeholder
    11 & Unidad 3: Estructuras de Datos - Binary Trees (BST, RBT, AVL) (fin) \\ % Placeholder
    12 & Unidad 4: Búsqueda básica - Breadth First Search (BFS) \\ % Placeholder
    13 & Unidad 4: Búsqueda básica - Depth First Search (DFS) \\ % Placeholder
    14 & Unidad 5: Optimización - Algoritmos voraces \\ % Placeholder
    15 & Unidad 5: Optimización - Programación dinámica (inicio) \\ % Placeholder
    16 & Unidad 5: Optimización - Programación dinámica (fin) / Repaso Final \\ % Placeholder
    17 & \textbf{Examen Final} \\ % Placeholder
\end{longtable}
\vspace{0.5cm}

% --- VIII. BIBLIOGRAFÍA ---
\section*{VIII. BIBLIOGRAFÍA}
\subsection*{Bibliografía Obligatoria} % Placeholder - Añadir libros específicos
\begin{itemize}[leftmargin=*]
    \item Cormen, T. H., Leiserson, C. E., Rivest, R. L., \& Stein, C. (2009). \textit{Introduction to Algorithms} (3rd ed.). MIT Press.
    \item Sedgewick, R., \& Wayne, K. (2011). \textit{Algorithms} (4th ed.). Addison-Wesley Professional.
\end{itemize}

\subsection*{Bibliografía Complementaria} % Placeholder - Añadir libros específicos
\begin{itemize}[leftmargin=*]
    \item Kleinberg, J., \& Tardos, É. (2005). \textit{Algorithm Design}. Addison-Wesley.
    \item Skiena, S. S. (2008). \textit{The Algorithm Design Manual} (2nd ed.). Springer.
\end{itemize}
\textit{Nota: Se utiliza el estilo de citación APA 7ma edición.}
\vspace{0.5cm}

% --- IX. ACERCA DEL DOCENTE (Opcional) ---
\section*{IX. ACERCA DEL DOCENTE} % Placeholder
\begin{itemize}[leftmargin=*]
    \item \textbf{Nombre del Profesor:} Dr. Juan Pérez
    \item \textbf{Correo Electrónico:} jperez@example.edu
    \item \textbf{Horario de Asesoría:} Miércoles 10:00 - 11:00
    \item \textbf{Formación:} Doctor en Ciencias de la Computación, Universidad de Stanford
    \item \textbf{Especialización:} Algoritmos de optimización y aprendizaje automático
\end{itemize}
\vspace{0.5cm}

% --- X. POLÍTICA CONTRA EL PLAGIO (Opcional pero recomendado) ---
\section*{X. POLÍTICA CONTRA EL PLAGIO}
\textit{El plagio y cualquier forma de deshonestidad académica serán sancionados de acuerdo con el reglamento de la institución. Todos los trabajos presentados deben ser originales. Se espera que los estudiantes citen adecuadamente todas las fuentes utilizadas. Para más detalles, consultar el reglamento estudiantil de la universidad.}

\end{document}
% --- END OF DOCUMENT ---